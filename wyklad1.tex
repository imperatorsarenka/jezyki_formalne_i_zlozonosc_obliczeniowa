\chapter{Wykład 12.10.2019}
\label{ch:wyklad1}

\section{Złożoność obliczeniowa}

Zagadnienia złożoności obliczeniowej
- jakie sa koszty prowadzenia obliczeń czasowe i pamięciowe:
\begin{itemize}
  \item Złożoność wykładnicza
  \item Nierozsądne gospodarowanie czasem
  \item Nierozsądne gospodarowanie pamięciową \ldots
\end{itemize}


\section{Gramatyka}

\newtheorem*{theorem*}{Gramatyka}
\begin{theorem*} Jak poprawnie budować wyrażenia danego języka (zbiór zasad).
Gramatyka inaczej jest nazywana syntaktyką albo składnią. \end{theorem*}



%schemat gramatyki
\begin{tikzpicture}

\node[draw, rectangle, minimum width = 3 cm, minimum height = 2 cm] (fl) at (0,0) { \bf JĘZYK};


% oznaczenie nad dgłównym prostokatem 
%\node[above] at (fl.north) {$V(B)$};

\draw[<-] (fl) -- node[above]{ \bf gramatyka} node[below]{generowanie wyrażeń} ++(-6,0);
\draw[<-] (fl) -- node[above]{\bf automat} node[below]{rozpoznawanie wyrażeń tego języka} ++(8,0);
\end{tikzpicture}


Między innymi kompilator posiada w sobie element rozpoznający gramatykę. \newline


%%%%%%%%%%%%%%%%%%%%%% Symbol %%%%%%%%%%%%%%%%%%%%


\section{Symbol a znaczenie symbolu}

 \subsection{Abstrakcyjne pojęcie liczby}


Warto odróżnić symbol od jego znaczenia. Np. liczbę dwa można zapisywac w postaci symoblu cyfry arabskiej { \bf 2} lub rzymskiej  { \bf II}. To samo dotyczy słowa { \bf słoń } - słowo oznacza wielkie kilkutonowe zwierze ale nim nie jest (nie jest bytem materialnym). 

\newtheorem*{theorem2*}{Abstrahować}
\begin{theorem2*} Abstrachować znacyz pomijać. Np.: abstrakcyjna liczba dwa powstała z pominięciem takich cech jak wielkość, pochodzenie.
 \end{theorem2*}
 
\subsection{Przykład powstania liczby}
Różna materialne nośniki niosące te same liczby obiektów o różnych cechach.Opisanie wspólnej cechy obiektów - { \bf liczebności} .

 
\begin{enumerate}[label=(\roman*)]
  \item {\bf couple} of people (para ludzi - 2)
  \item {\bf pair} of pistols (para pistoletów - 2)
  \item {\bf yoke} of oxen (zaprzęg dwa zwięrzęta)
\end{enumerate}
 
 

Abstakcyjna liczba { \bf 2} powstała abstrahując od pochodzenia (np. zwierzęcia), wielkości (np. broni) czy płci (para ludzi) pozostawiając tylko jedną wspólną cechę, którą jest { \bf liczebność} . 
 

%%%%%%%%%%%%%%%%%%%%%% języki formalne %%%%%%%%%%%%%%%%%%%%

\section{Języki formalne}
\subsection{Pojęcia}

Ciągi i zbiory ciągów traktowane są jako obiekty materialne a { \bf nie } abstrakycjne. \newline
{ \bf Skończoność} - ważna cecha alfabetu/zbioru ponieważ tylko skończone zbiory danych można przechowywać w { \bf fizycznym urządzeniu}. 

\newtheorem*{theorem3*}{Alfabet V}
\begin{theorem3*} Alfabet V to: { \bf dowolny} , { \bf niepusty} , { \bf skończony zbior znaków}  \newline np.: V = \{I\} , V' = \{a,b\}.
\newline


 \end{theorem3*}


\newtheorem*{theorem4*}{Słowo nad alfabetem V}
\begin{theorem4*} Słowo nad alfabetem V to dowolny, skończony ciąg znkaów z V. np.: {\bf IIII} (słowo nad alfabete V=\{I\}) czy {\bf abba} (słowo nad alfabetem V=\{a,b\}) 
 \end{theorem4*}


\newtheorem*{theorem5*}{Słowo puste $\epsilon$}
\begin{theorem5*}Słowo puste $\epsilon$ - słowo o 0 (zerowym) wystąpieniu symboli. Uwaga! Spacja {\bf NIE} jest słowem pustym.
 \end{theorem5*}
 
 
 \newtheorem*{theorem6*}{V*}
\begin{theorem6*}Zbiór wszsytkich słów nad alfabetem V. Łącznie z pustym słowem $\epsilon$.
 \end{theorem6*}
 
 
\newtheorem*{theorem7*}{V* \textbackslash \{$\epsilon$\} = V+}
\begin{theorem7*}
Zbiór wszsytkich niepustych słów. (ŁWyłączenie ze zbioru pustego słowa $\epsilon$)
 \end{theorem7*}
 
\newtheorem*{theorem8*}{Oznaczenie słów}
\begin{theorem8*}
Słowa oznaczane są wielkimi literami z końca alfabetu łacińskiego, np.: { \bf P},{\bf Q},{\bf R}. 
 \end{theorem8*}
 

\subsection{Operacja konkatenacji}
\subsubsection{Własności konkatenacji}
\begin{itemize}
  \item Konkatenacja jest działaniem łacznym w zbiorze słów
  \item Konkatenacja w ogólnoście {\bf NIE} jest przemienna (bywa przemienna dla tyh samych słów {\bf ab} {\bf ab } ) lub jeśli alfabet skada sie tylko z jednego znaku np V = \{a\}
  \item $\epsilon$ słowo puste zachowje się jak element neutralny dla operacji konkatenacji: \newline $\epsilon$P $\subset$ 
  P$\epsilon$ = P.
\end{itemize}


\subsection{Konkatnacja a grupa algebraiczna}

