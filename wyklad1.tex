\chapter{Wykład 12.10.2019}
\label{ch:wyklad1}

%%%%%%%%%%%%%%%%%%%%%%%%%%%%%%% Złożoność obliczeniowa %%%%%%%%%%%%%%%%%%%%%%%%%%%%%%%
\section{Złożoność obliczeniowa}

Zagadnienia złożoności obliczeniowej
- jakie sa koszty prowadzenia obliczeń czasowe i pamięciowe:
\begin{itemize}
  \item Złożoność wykładnicza
  \item Nierozsądne gospodarowanie czasem
  \item Nierozsądne gospodarowanie pamięciową \ldots
\end{itemize}


%%%%%%%%%%%%%%%%%%%%%%%%%%%%%%% Gramatyka %%%%%%%%%%%%%%%%%%%%%%%%%%%%%%%
\section{Gramatyka}

\newtheorem*{theorem*}{Gramatyka}
\begin{theorem*} Jak poprawnie budować wyrażenia danego języka (zbiór zasad).
Gramatyka inaczej jest nazywana syntaktyką albo składnią. \end{theorem*}



%schemat gramatyki
\begin{tikzpicture}

\node[draw, rectangle, minimum width = 3 cm, minimum height = 2 cm] (fl) at (0,0) { \bf JĘZYK};


% oznaczenie nad dgłównym prostokatem 
%\node[above] at (fl.north) {$V(B)$};

\draw[<-] (fl) -- node[above]{ \bf gramatyka} node[below]{generowanie wyrażeń} ++(-6,0);
\draw[<-] (fl) -- node[above]{\bf automat} node[below]{rozpoznawanie wyrażeń tego języka} ++(8,0);
\end{tikzpicture}


Między innymi kompilator posiada w sobie element rozpoznający gramatykę. \newline


%%%%%%%%%%%%%%%%%%%%%%%%%%%%%%% Symbol %%%%%%%%%%%%%%%%%%%%%%%%%%%%%%%


\section{Symbol a znaczenie symbolu}

 \subsection{Abstrakcyjne pojęcie liczby}


Warto odróżnić symbol od jego znaczenia. Np. liczbę dwa można zapisywac w postaci symoblu cyfry arabskiej { \bf 2} lub rzymskiej  { \bf II}. To samo dotyczy słowa { \bf słoń } - słowo oznacza wielkie kilkutonowe zwierze ale nim nie jest (nie jest bytem materialnym). 

\newtheorem*{theorem2*}{Abstrahować}
\begin{theorem2*} Abstrachować znacyz pomijać. Np.: abstrakcyjna liczba dwa powstała z pominięciem takich cech jak wielkość, pochodzenie.
 \end{theorem2*}
 
\subsection{Przykład powstania liczby}
Różna materialne nośniki niosące te same liczby obiektów o różnych cechach.Opisanie wspólnej cechy obiektów - { \bf liczebności} .

 
\begin{enumerate}[label=(\roman*)]
  \item {\bf couple} of people (para ludzi - 2)
  \item {\bf pair} of pistols (para pistoletów - 2)
  \item {\bf yoke} of oxen (zaprzęg dwa zwięrzęta)
\end{enumerate}
 
 

Abstakcyjna liczba { \bf 2} powstała abstrahując od pochodzenia (np. zwierzęcia), wielkości (np. broni) czy płci (para ludzi) pozostawiając tylko jedną wspólną cechę, którą jest { \bf liczebność} . 
 

%%%%%%%%%%%%%%%%%%%%%%%%%%%%%%% języki formalne %%%%%%%%%%%%%%%%%%%%%%%%%%%%%%%

\section{Języki formalne}
\subsection{Pojęcia}

Ciągi i zbiory ciągów traktowane są jako obiekty materialne a { \bf nie } abstrakycjne. \newline
{ \bf Skończoność} - ważna cecha alfabetu/zbioru ponieważ tylko skończone zbiory danych można przechowywać w { \bf fizycznym urządzeniu}. 

\newtheorem*{theorem3*}{Alfabet V}
\begin{theorem3*} Alfabet V to: { \bf dowolny} , { \bf niepusty} , { \bf skończony zbior znaków}  \newline np.: V = \{I\} , V' = \{a,b\}.
\newline


 \end{theorem3*}


\newtheorem*{theorem4*}{Słowo nad alfabetem V}
\begin{theorem4*} Słowo nad alfabetem V to dowolny, skończony ciąg znkaów z V. np.: {\bf IIII} (słowo nad alfabete V=\{I\}) czy {\bf abba} (słowo nad alfabetem V=\{a,b\}) 
 \end{theorem4*}


\newtheorem*{theorem5*}{Słowo puste $\epsilon$}
\begin{theorem5*}Słowo puste $\epsilon$ - słowo o 0 (zerowym) wystąpieniu symboli. Uwaga! Spacja {\bf NIE} jest słowem pustym.
 \end{theorem5*}
 
 
 \newtheorem*{theorem6*}{V*}
\begin{theorem6*}Zbiór wszsytkich słów nad alfabetem V. Łącznie z pustym słowem $\epsilon$.
 \end{theorem6*}
 
 
\newtheorem*{theorem7*}{V* \textbackslash \{$\epsilon$\} = V+}
\begin{theorem7*}
Zbiór wszsytkich niepustych słów. (Wyłączenie ze zbioru pustego słowa $\epsilon$)
 \end{theorem7*}
 
\newtheorem*{theorem8*}{Oznaczenie słów}
\begin{theorem8*}
Słowa oznaczane są wielkimi literami z końca alfabetu łacińskiego, np.: { \bf P},{\bf Q},{\bf R}. 
 \end{theorem8*}


%%%%%%%%%%%%%%%%%%%%%%%%%%%%%%%
\newpage
\section{Konkatenacja}
\subsection{Operacja konkatenacji}

\newtheorem*{theorema*}{Konkatenacja dwóch słów}
\begin{theorema*}
Konkatenacją dwóch słów {\bf P} i {\bf Q} nazywamy słowo {\bf PQ} zdefiniowane w następujący sposób:

\begin{enumerate}[label=(\roman*)]
  \item jeżeli {\bf P}=$a_{1}, ... ,a_{n}$ gdzie {\bf a}=$b_{1}, ... ,b_{n}$ n,m $\ge$ 0 to {\bf PQ}= $a_{1},...,a_{n}b_{1},...,b_{n}$
  \item Jeżeli {\bf P}=$\epsilon$, to {\bf PQ}=Q.\newline 
  Alternatywnie to {\bf Q}=$\epsilon$ i wtedy {\bf PQ}=P. \newline
  Gdy {\bf P}={\bf Q}=$\epsilon$ to {\bf PQ=}$\epsilon\epsilon$ = $\epsilon$ . 
\end{enumerate} 

 
 \end{theorema*}


%%%%%%%%%%%%%%%%%%%%%%%%%%%%%%%

\subsubsection{Własności konkatenacji}
\begin{itemize}
  \item Konkatenacja jest działaniem łacznym w zbiorze słów
  \item Konkatenacja w ogólnoście {\bf NIE} jest przemienna (bywa przemienna dla tyh samych słów {\bf ab} {\bf ab } ) lub jeśli alfabet skada sie tylko z jednego znaku np V = \{a\}
  \item $\epsilon$ słowo puste zachowje się jak element neutralny dla operacji konkatenacji: \newline $\epsilon$P $\subset$ 
  P$\epsilon$ = P.
\end{itemize}

%%%%%%%%%%%%%%%%%%%%%%%%%%%%%%%

\subsection{Konkatenacja NIE jest grupą algebraiczną  $\heartsuit$}
Pomimo abstrakcyjnego znaczenia liczb, ich mentalna reprezentacja jest jednak w urządzeniu czymś materialnym (stanami wysokich i niskich napięć).

 $V^{*}$ - zbiór wszystkich elementow (słów) nad alfabatem { \bf V} (łącznie z elementem pustym $\epsilon$)

$\circ$ - oznacza działanie w grupie 

{\bf e} - litera e jest symbolem elementu neutralnego \newline

Przykład łączności:
a) dodawanie np. : 2 + (3 + 5) = (2 + 3) + 5
b) mnożenie np.:  2 * (3 * 5) = (2 * 3) * 5
 
Konkatenacja jest grupą (z algebry) jeśli spełnia warunki na bycie grupą:


\begin{enumerate}[label=(\roman*)]
  \item operacja $\circ$ jest łączna w grupie;
  \item $\exists e$, $\forall x $ Istnieje element neutralny dal każdego x, taki że  $x \circ e = e \circ x = x$;
  \item Dla każdego x $\forall x $   Istnieje element odwrotny  $\exists x^{-1}$, taki, że  $x \circ  x^{-1} = x^{-1} \circ x = e$. \newline Warunek nie spełniony przez konkatenację - nie istnieje w ogólności takie słowo gdzie: słowo + słowo$^{-1} = \epsilon$ 
  (szczególny przypadek spełnienia to $\epsilon$ + $\epsilon$ = $\epsilon$, bo element neutralny jest sam do siebie odwrotny $\epsilon^{-1} = \epsilon$ ) \newline $\Rightarrow$ {\bf WARUNEK NIE JEST W OGÓLNOŚCI SPEŁNIONY - konkatenacja NIE jest grupą!}
\end{enumerate}

%%%%%%%%%%%%%%%%%%%%%%%%%%%%%%%

\subsection{Podsłowo słowa}
Zbiór { \bf A } $\subset$   { \bf A } i analogicznie  { \bf abca } $\sqsubset$  { \bf abca }
(Znak $\sqsubset$  to taka kanciasta inkluzja oznaczenie używane przy słowach)


\newtheorem*{theorem9*}{Podsłowo}
\begin{theorem9*}
Mówimy, że słowo {\bf Q} jest podsłowem słowa {\bf P} wtedy i tylko wtedy gdy, istnieją słowa $Q_{1}$ i $Q_{2}$ takie, że:
\begin{center}
$P = Q_{1}${\bf Q}$Q_{2}$.
\end{center}
Np:. słowo {\bf bc} jest podsłowem słowa {\bf abcd} 


\begin{center}
\begin{tikzpicture}
\node[left]{$Q_{1}$};
\draw[->] (0,0)-- (1.9,0.8) node[right]{{\bf a}$\underbrace{bc}_{Q}${\bf d}};
\draw[<-] (3,0.8) -- (5,0) node[right]{$Q_{2}$};
\end{tikzpicture}
\end{center}

Widać, tutaj, zę Q to słowo {\bf ab}, $Q_{1} = a$, $Q_{2}=d$
\end{theorem9*}
 

\newtheorem*{theorem10*}{Prefix słowa}
\begin{theorem10*}
Słowo $Q$ jest prefixem słowa $P$ jeśli $P = QQ_{1}$.
\end{theorem10*}
 
\newtheorem*{theorem11*}{Suffix słowa}
\begin{theorem11*}
Słowo $Q$ jest suffixem słowa $P$ jeśli $P =Q_{1}Q$.
\end{theorem11*}

\newtheorem*{theorem12*}{Infix słowa}
\begin{theorem12*}
Słowo $Q$ jest infixem słowa $P$ jeśli $P =Q_{1}QQ_{2}$  \newline
gdzie $Q_{1}\neq \epsilon$ i gdzie $Q_{2} \neq \epsilon$.
\end{theorem12*}


%%%%%%%%%%%%%%%%%%%%%%%%%%%%%%%

\subsection{Długość słowa}
\newtheorem*{theorem13*}{Długość słowa $\mid P\mid$}
\begin{theorem13*}
Długością słowa $P \subset V^{*}$ nazywamy liczbę naturalną $\mid P\mid$ definiujemy w sposób indukcyjny:
 
\begin{enumerate}[label=(\roman*)]
  \item $ \mid  \epsilon \mid=0$
  \item $ \mid P{ \bf a}  \mid= \mid P \mid + 1$; gdzie P to ciąg (słowo) a {\bf a} to symbol (dodatkowa litera w słowie).
\end{enumerate} 
 
\end{theorem13*}


\paragraph{Przykład 1.}
Długość słowa {\bf abc} \newline
$\mid abc \mid = \mid ab \mid + 1 = ( \mid a \mid + 1) + 1 = 
(\mid \epsilon a \mid + 1) + 1 =   ((\mid \epsilon \mid + 1 )+1)+1 =
((0+1)+1)+1 = 3
$ 

\paragraph{Przykład 2.}
Obliczyć ilość wszystkich podsłów słowa {\bf P} gdy dana jest długość słowa $\mid P \mid = 4$ \newline
Odp: { \bf NIE WIĘCEJ NIŻ} 11. \newline
Rozwiązanie a):\newline
Weźmy dla przykładu słowo {\bf abcd }. 

Podsłowo {\bf abcd} $\subset$ {\bf abcd} - jedno podsłowo długości 4. (Słowo jest samo swoim podsłowem; kolejnosć znaków też ma znaczenie np słowo {\bf bcda} $\not\subset $ {\bf abcd}  ).
\newline

Podsłowa długości 3 (2 takie słowa)
{\bf abc} $\subset$ {\bf abcd},
{\bf bcd} $\subset$ {\bf abcd};

Podsłowa długości 3 (2 takie słowa)
{\bf ab} $\subset$ {\bf abcd},
{\bf bc} $\subset$ {\bf abcd},
{\bf cd} $\subset$ {\bf abcd};

Podsłowa długości 1 (4 takie słowa)
{\bf a} $\subset$ {\bf abcd},
{\bf b} $\subset$ {\bf abcd},
{\bf c} $\subset$ {\bf abcd},
{\bf d} $\subset$ {\bf abcd};

Odp a) Dla słowa { \bf abcd} mamy (1+2+3+4) + 1( dodajemy jeden bo znak pusty $\epsilon$ )

Rozwiązanie b):
Załóżmy, że szukamy wszystkich podsłów słowa P = {\bf aaaa}. 
aaaa $\subset$  aaaa (1 podsłowo)
aaa $\subset$  aaaa (1 podsłowo)
aa $\subset$  aaaa (1 podsłowo)
a $\subset$  aaaa (1 podsłowo)

Odp a) Dla słowa { \bf aaaa} mamy (1+1+1+1) + 1( dodajemy jeden bo znak pusty $\epsilon$ ), ponieważ {\bf NIE ROZRÓŻNIAMY ZNAKÓW MIĘDZY SOBĄ tzn.: zawsze a == a} (nie rozróżniamy permutacji tych samych elementów)

%%%%%%%%%%%%%%%%%%%%%%%%%%%%%%% Zadanie domowe 12.10.2019 %%%%%%%%%%%%%%%%%%%%%%%%%%%%%%%

\begin{tcolorbox}
	\textbf{Zadanie Domowe 12.10.2019} \newline
	
	Napisać procedurę w pseudokodzie:
	Dla zadanego słowa długośc {\bf n} napisac procedurę, które pokaże liczbę {\bf k}
	podsłów oraz wypisze całą listę konkretnych podsłów.
\end{tcolorbox}

%%%%%%%%%%%%%%%%%%%%%%%%%%%%%%% Rozwiazanie %%%%%%%%%%%%%%%%%%%%%%%%%%%%%%%
\begin{lstlisting}


Tu bedzie rozwiazanie:
    if (n == 0 || n == 1){    
      return n;        
    }        
    j = 0;    
    for (i = 0; i < n-1; i++){      
      if (arr[i] != arr[i+1]){        
        arr[j] = arr[i];       
        j++;      
      }       
    }      
    arr[j++] = arr[n-1];
\end{lstlisting}

%%%%%%%%%%%%%%%%%%%%%%%%%%%%%%% Długość konktatenacji słów %%%%%%%%%%%%%%%%%%%%%%%%%%%%%%%


\subsection{Długość konkatenacji słów}

\newtheorem*{theorem14*}{Długość konktatenacji słów}
	\begin{theorem14*}
	\begin{center}
	$\mid PQ \mid = \mid P \mid + \mid Q \mid$
	\end{center}
	
	Dowód (indukcja matematyczna):\newline
	$\forall P,Q: \mid PQ \mid = \mid P \mid + \mid Q \mid$ (Indukcja po długości $\mid Q \mid$)
	
	\begin{enumerate}[label=(\roman*)]
		\item $\mid Q \mid = \epsilon$ \newline
		LewaStrona: $\mid PQ \mid = \mid P \epsilon \mid = \mid P \mid$ gdzie k $ \leq $ n \newline
		PrawaStrona: $\mid P \mid + \mid Q \mid = 
		\mid P \mid + \mid \epsilon \mid = 
		\mid P \mid+ 0 = \mid P \mid $ = { \bf LewaStrona} c.d.n \newline
		  
		  
		\item  $\lambda$: $\mid PQ \mid = \mid P \mid + \mid Q \mid $ \newline
		PrawaStrona: $\mid P(Qa) \mid = \mid P \mid  + \mid Qa \mid$  gdzie n=0\newline
		LewaStrona: $ \mid P(Qa) \mid \overset{\mathrm{jeżeli}}{=} 
		\mid (PQ)a \mid \overset{\mathrm{ii}}{=} 
		\mid PQ \mid + 1 \overset{\mathrm{\lambda}}{=}$ \newline
		$(\mid P \mid + \mid Q \mid) + 1 = 
		\mid P \mid + (\mid Q \mid + 1) = 
		\mid P \mid + \mid Qa \mid$ ,..., = { \bf  PrawaStrona } c.n.d
	
	\end{enumerate} 
\end{theorem14*}


%%%%%%%%%%%%%%%%%%%%%%%%%%%%%%% N-ta potęga słowa %%%%%%%%%%%%%%%%%%%%%%%%%%%%%%%

\subsection{N-ta potęga słowa}
Przykład potęgowania liczb: \textbf{$ \underbrace{7*7* ... *7}_{n-razy} = 7^{n}$}\newline
Potęgowanie słów: \textbf{abcabcabc} = $(abc)^{3} /neq a^{3}b^{3}c^{3}$

\newtheorem*{theorem15*}{N-ta potęga liczby}
\begin{theorem15*}
	Potęgowanie liczb - widać że pierwszy warunek $a^{0} = 1$ jest stworzony przez
	 matematyków, niejako sztucznie ale jest to wymagane dla prawdiłowości działania
	 indukcji matematycznej; a w praktyce jest to często warunek zatrzymania funkcji
	 obliczającej np. silnie iteracyjnie czy rekurencyjnie.

	\begin{enumerate}[label=(\roman*)]
		\item $a^{0} = 1$
		
		\item $a^{n+1} = a^{0+1}=a^{0}*a$
	\end{enumerate} 
\end{theorem15*}


Podobnie jak liczby potęguje się też słowa, ale ze względu na swoją specyfikę kolejność znaków w słowie podczas mnożenia ma znaczenie!

\newtheorem*{theorem16*}{N-ta potęga słowa}
\begin{theorem16*}
	N-tą potęgą słowa \textbf{P}, oznaczamy \textbf{$P^{n}$}, nazywamy słowo 
	zdefiniowane indukcyjnie w następujący sposób:

	\begin{enumerate}[label=(\roman*)]
		\item $P^{0} = \epsilon$
		\item $P^{n+1} = P^{n}P$
	\end{enumerate} 

Dowód:
$\underbrace{P^{1}}_{}= P^{0+1} \overset{\mathrm{ii}}{=}  P^{0}\underbrace{P}$

\end{theorem16*}

Przykład:\newline
{\color{red} $a^{3}(ba)^{2}= aaababa$ } \newline
{\color{red} $(abc)^{3} \neq a^{3}b^{3}c^{3}$}


%%%%%%%%%%%%%%%%%%%%%%%%%%%%%%% Rewers %%%%%%%%%%%%%%%%%%%%%%%%%%%%%%%

\subsection{Operacja Odbicia zwierciadlanego (rewers) słowa P}
Przykład: \newline
$(abc)^{-1}$ = cba

\newtheorem*{theorem17*}{Odbicie zwierciadlane słowa P}
\begin{theorem17*}
Odbicie zwierciadlane słowa \textbf{P}, oznaczamy przez $P^{-1}$ (P prim)i definiujemy je indukcyjnie po $\mid P \mid '$ w następujący sposób:

	\begin{enumerate}[label=(\roman*)]
		\item $\epsilon^{-1} = \epsilon$
		\item $(Pa)^{-1} = aP^{-1}$  //odwróciliśmy a teraz odwracamy resztę słowa 			
				\textbf{P}
	\end{enumerate} 

\end{theorem17*}


%%%%%%%%%%%%%%%%%%%%%%%%%%%%%%% Własności konkatenacji %%%%%%%%%%%%%%%%%%%%%%%%%%%%%%%

\subsection{Własności konkatenacji}
	\subsubsection{Rewers Konkatenacji (odbicie zwierciadlane)}
	$(PQ)^{-1} = Q^{-1}P^{-1}$

	\subsubsection{Rewers N-tej potęgi konkatenacji}
	$(P^{n})^{-1} = (P^{-1})^{n}$  np.: $((abc)^{3})^{-1} = (cba)^{3}$
	
	\subsubsection{Złożenia odbić zwierciadlanych konkatenacji}
	$(P^{-1})^{-1} = P$ //odbicia znoszą się



%%%%%%%%%%%%%%%%%%%%%%%%%%%%%%%%%%%%%%%%%%%%%%%%%%%%%%%%%%%%%%%%%%%%%%%%%%%%%%%%%%%%%%%%%

\section{Język}
{\color{red} Uwaga - alfabet V $\not\Leftrightarrow$ język.} \newline
Należy takze pamiętać, że:
 $V^{*}$ - zbiór wszystkich elementow (słów) nad alfabatem { \bf V} (łącznie z elementem pustym $\epsilon$)\newline \newline


%schemat gramatyki
\begin{tikzpicture}
\node[draw, rectangle, minimum width = 3 cm, minimum height = 2 cm] (fl) at (0,0) { \bf JĘZYK};


% oznaczenie nad dgłównym prostokatem 
%\node[above] at (fl.north) {$V(B)$};

\draw[<-] (fl) -- node[above]{ \bf generator}  ++(-6,0);
\draw[<-] (fl) -- node[above]{\bf rozpoznawanie} ++(8,0);
\end{tikzpicture}\newline

Przykładowo mamy \textbf{język angielski} składający się z małych liter \textbf{V=\{a,b,...z\}}
Słowo \textbf{cat} należy do tego języka  tzn.: \textbf{ cat $\subset$ jezykAngielki}; \newline 
a słowo \textbf{kot} chociaż jego znaki należą do alfabetu \textbf{V} to jednak słowo \textbf{"kot"} nie należy do \textbf{języka angielkiego} tzn.: \textbf{kot $\not\subset$ językAngielski}.

%%%%%%%%%%%%% rysunek zbioru %%%%%%%%%%%%%
\begin{center}
	\begin{tikzpicture}
		\draw (0,0) rectangle (8,5);  
		\node [at={(1,4)}] {$V^{*}$};
		\draw[pattern=north east lines, pattern color=light-gray ] 
				(4.5,2) ellipse[x radius = 3, y radius = 1.5];
		\node [at={(2.5,1.5)}] {kot};
		\draw [fill, color=black] (2.5,1.7) circle(0.05cm); %punkt
		\draw [fill, color=white] (5,2) ellipse [x radius =2, 
		y radius = 1];
		\node [at = {(5,2)}] {$cat$};
		\draw [fill, color=black] (5,2.2) circle(0.05cm); %punkt

	\end{tikzpicture}
\end{center}


\newtheorem*{theorem18*}{Język}
\begin{theorem18*}
	Językiem nad alfabetem \textbf{V} nazywamy dowolny podzbiór zbioru wszystkich słów nad \textbf{V} tj.:
	\begin{center}
		L $\subset V^{*}$ 
	\end{center}

\end{theorem18*}

\begin{tcolorbox}
	\textbf{Przykład} \newline
	Dany jest alfabet binary V=\{0,1\}, czyli, żeby zaprezenetować wszystie liczby parzyste większe od 0 nalezy na ostatnim bicie umieścić cyfrę 0.
	
	%%%%%%%%%%%%% rysunek zbioru liczby binarne %%%%%%%%%%%%%
	\begin{center}
		\begin{tikzpicture}
			\draw[pattern=north east lines, pattern color=light-gray ] 
					(4, 4) circle(2.3cm);
			\node [at={(3,5)}] {V*};
			\draw [fill, color=white] (5,3) circle(0.7cm);
			\node [at = {(5,3)}] {1,..,0};
		\end{tikzpicture}
	\end{center}
	
\end{tcolorbox}

%%%%%%%%%%%%%%%%%%%%%%%%% Język jako zbiór  %%%%%%%%%%%%%%%%%%%%%%%%%
\subsection{Język jako zbiór - własności}
	Jako obiekt język to zbiór $\rightarrow$ dziedziczy wszystkie własności zbioru:
	
	\begin{itemize}
	  \item A $\subset$ A
	  \item $V^{*} \subset V^{*}$
	  \item $\O \subset$ A
	  \item $\O \subset V^{*}$
	\end{itemize}
	
	
	\begin{tcolorbox}
	
		\textbf{Przykład} \newline
		%%%%%%%%%%%%% zbiory %%%%%%%%%%%%%
		\begin{center}
			\begin{tikzpicture}
				\draw[pattern=north east lines, pattern color=light-gray ] 
						(4, 4) circle(2.3cm);
				\node [at={(3,5)}] {V*};
				\draw [fill, color=white] (4,3) circle(1.2cm);
				\node [at = {(4,3)}] {L1};
			\end{tikzpicture}
		\end{center}
	
		Przykładowe operacje
		\begin{itemize}
		  \item $L_{1}, L_{2} \subset$ $V^{*}$
		  \item $L_{1} \cup L_{2}$
		  \item $L_{1} \cap L_{2}$
		  \item $L_{1} \setminus L{2}$
		  \item $\overline{L_{1}} = V^{*} \setminus L_{1}$
		\end{itemize}
	

\end{tcolorbox}


%%%%%%%%%%%%%%%%%%%%%%%%% Konkatenacja języków  %%%%%%%%%%%%%%%%%%%%%%%%%
\subsection{Konkatenacja języków}

\newtheorem*{theorem19*}{Konkatenacja języków}
\begin{theorem19*}
Dla języków $L_{1}$ i $L_{2}$ przez konkatenację tych języków, oznaczoną $L_{1}L_{2}$ rozumiemy zbiór:=
	\begin{center}
		$L_{1}L_{2} =\{P_{1}P{2}: P_{1} \in L_{1}, P_{2} \in L_{2}\}$
	\end{center}
\end{theorem19*}


\begin{tcolorbox}
	\textbf{Przykład 1} \newline
	Mając języki  $L_{1}=\{a, ab, \epsilon \}$, $L_{2}=\{ b,ab, ba \}$.\newline
	Ile możemy utworzyć słów poprzez konkatenację języków $L_{1}$ i $L_{2}$? \newline
	\textbf{Odp: Maksymalnie 9.} \newline
	Rozwiązanie (najlepiej takie rzeczy robić tabelką):\newline
		\begin{tabular}{|l||*{5}{c|}}\hline
			\backslashbox{$L_{1}$}{$L_{2}$}
			&\makebox[3em]{b}&\makebox[3em]{ab}&\makebox[3em]{ba}
			\\\hline\hline
			a  			& ab		& $a^{2}b$		& $ba^{2}$ 		\\\hline
			ab 			& $ab^{2}$ 	& $(ab)^{2}$	& $ab^{2}a$ 	\\\hline
			$\epsilon$ 	& b			& ab			& ba			\\\hline
		\end{tabular}\newline
	Tabelka jest idealnym sposobem liczenia konkatenacji języków. \newline
	{\color{red} Uwaga przy konkatenacji trzeba uważać, cyz pytanie dotyczyło wszystkich mozliwych kombinacji( wtedy liczymy z powtórzeniami), czy tylko unikatowych słów.} \newline

\end{tcolorbox}


\begin{tcolorbox}
	\textbf{Przykład 2} \newline
	Mając języki: \newline 
		$L_{1}= \{ \epsilon, a, a^{2}, a^{3}, a^{4}, ... \} = {a^{n}: n \ge 0}$ \newline
		$L_{2}= \{ \epsilon, b, b^{2}, b^{3}, b^{4}, ... \} = {b^{n}: n \ge 0}$ \newline

	\textbf{Odp:
		$L_{1}L_{2} = \{ a^{n}b^{m}: n,m \ge 0 \}$}	// n,m $\ge$ 0 bo dopuszczamy $\epsilon * \epsilon równy 0$
	

\end{tcolorbox}

%%%%%%%%%%%%%%%%%%%%%%%%%%%%%%%%%%%%%%%%%%%%%%%%%%%%%%%%%%%%%%%%%%%%%%%%%%%%%%%%%%%%%%%%%%%%%%%%%%%%%%%%%%%%%%%%%%%%%%%%%%%%%%%%%%%%%%%%%%%%%%%%%%%%%%%%%%%%%%%%%%%%%%%%%%%%%%%%%%%%%%%%%%%%%%%%%%%%%%%%%%%%%%%%%%%%%%%%%%%%%%%%%%%%%%%%%%%%%%%%%%%%%%%%%%%%%%%%%%%%%%%%%%%%%%%%%%%%%%%%%%%%%%%%%%%%%%%%%%%%%%%%%%%%%%%%%%%%%%%%%%%%%%%%%%%%%%%%%%%%%%%%%%%%%%%%%%%%%%%%

\newpage
\chapter{Brudnopis}
\label{ch:brudnopis}

$$\heartsuit$$

$\underbrace{wyr1}_{wyr2},\overbrace{wyr1}^{wyr2}$

$\exists y\geq 0\;\forall x:x\leq y$

 $\sqsupset$ 
 
 
 \begin{tikzpicture}
\draw[->] (0,0) node[left]{Q} -- (2,1) node[right]{abcd} -- (3,1)(5,0) node[right]{Q2};


\end{tikzpicture}


\xymatrix{
    A \ar[d] \ar[dr] \ar[drr] &   &   \\
    B                         & C & D }

\xymatrix{A &*+[F]{\sum_{i=n}^m {i^2}} \\& {\bullet} & D \ar[ul]     }

\xymatrix{A \ar[d]^b \ar[r]^a &B\ar[d]^c\\C \ar[r]^d          &D}



\begin{picture}(300,20)
\put(50,10){\vector(2,-3){20}}
\put(50,10){\vector(2,-1){60}}
\put(-10,15){these guys \textbf{here} must be eliminated. The next one \textbf{here} too, of course.}
\put(230,10){\vector(-2,-3){20}}
\end{picture}
\[ \bigl\{ \underbracket[0.5pt]{x \in \mathbb{Z}}_{\clap{\footnotesize Menge aller $x$ aus $\mathbb{Z}$}} \stackrel{\Shortstack {\clap{\footnotesize mit der Eigenschaft}\\$\downarrow $}}{\mid} -1 \le x \le 2\bigr\} \]
\\[1cm]
And I want to get it like this:
\[ \{x \in \mathbb{Z} \ | \ -1 \le x \le 2 \}, \]
where the \verb|\overbrace| doesn't wait till the \verb|\underbrace| is going to have the work done.
